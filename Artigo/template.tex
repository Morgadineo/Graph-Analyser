%========Template.tex=========================================================%
% Arquivo com as configurações do template utilizado.
%=============================================================================%


%==> Pacotes necessários <====================================================%
\usepackage{amsmath}
\usepackage{amssymb}  % Símbolos matemáticos
\usepackage{graphicx} % Imagens
\usepackage{color}
\usepackage{multirow}
\usepackage{booktabs}
\usepackage{fancyhdr}
\usepackage{listings} % Trechos de código
\usepackage{xcolor}

\graphicspath{ {./Imagens/} }

%==> Configurações de layout <================================================%
\headsep=5mm \topmargin=0cm \oddsidemargin=-0.5cm
\evensidemargin=-0.5cm \textheight=23.1cm \textwidth=17.5cm \footskip=8mm
\columnsep=7mm \setlength{\doublerulesep}{0.1pt}
\footnotesep=3.5mm \arraycolsep=2pt
\font\tenrm=cmr10

%========Formatação do título e autor=========================================%
\def\title#1{\vspace{3mm}\begin{flushleft}\Large\bf\boldmath #1 \end{flushleft}\vspace{1mm}}
\def\author#1{\begin{flushleft}\normalsize #1\end{flushleft}\vspace*{-4pt}}

%==> Cabeçalhos e rodapés <===================================================%
\pagestyle{fancy}
\fancyhead[LO]{\small\sl Trabalho de Teoria dos Grafos. Grupo D}
\fancyhead[RO]{\small\thepage}
\fancyhead[LE]{\small\thepage}
\fancyhead[RE]{\small\sl Universidade Vila Velha}

%==> Outras configurações do template <=======================================%
\renewcommand\tablename{\bf \footnotesize Table}
\renewcommand\figurename{\footnotesize Fig.\!\!}

%==> Formatação de código <===================================================%
\renewcommand{\lstlistingname}{Código}

\lstset{
    language=Python,
    basicstyle=\ttfamily\footnotesize, % Fonte do código
    keywordstyle=\color{blue}, % Cor das palavras-chave
    stringstyle=\color{green}, % Cor de strings
    commentstyle=\color{gray}, % Cor de comentários
    numbers=left, % Numeração de linhas
    numberstyle=\tiny, % Estilo dos números
    stepnumber=1, % Frequência da numeração
    frame=single, % Moldura ao redor do código
    breaklines=true, % Quebra de linha automática
    tabsize=4, % Tamanho do tab
}

