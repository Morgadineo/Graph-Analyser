%=======> introducao.tex <====================================================%
% Arquivo contendo a sessão de introdução.


\section{Introdução}
A disciplina de Teoria dos Grafos, ministrada pelo professor Dr.Jairo Lucas de Moraes, propôs um projeto prático com o 
objeto de desenvolvermos um programa, na linguagem Python ou C, capaz de analisar um grafo direcionado qualquer e 
retornar algumas informações sobre o mesmo. O presente artigo tem como finalidade descrever e detalhar a abordagem 
utilizada pelo grupo para a implementação das funcionalidades solicitadas, bem como discutir sobre as estruturas de 
dados empregadas. Como proposta do projeto, o programa deveria cumprir as seguintes exigências:
\begin{itemize}
	\item Receber um grafo direcionado qualquer e:
	\item representá-lo nos formatos de matriz de adjacência e lista de adjacência;
	\item verificar a existência de laços;
	\item identificar suas características (simples, completo);
	\item analisar se o mesmo é uma árvore e caso seja, identificar seu tipo.
\end{itemize}

Com os requisitos definidos, é possível partir para a primeira etapa da implementação do programa. Quando se pretende 
implementar algoritmos computacionais para resolução de um problema, especialmente para problemas matemáticos, é de 
extrema importância estar bem fundamentado com as bases relacionadas ao conteúdo do problema. Muitas soluções 
algorítmicas podem ser simplificadas utilizando fundamentos matemáticos. Um exemplo de problema da qual a abordagem 
matemática simplifica a complexidade, é o problema de encontrar a soma dos \textit{n} primeiros números $\mathbb{N}$. A 
abordagem mais intuitiva, que utiliza uma estrutura de repetição iterando sob o intervalo e somando os números
percorridos, possui complexidade \textbf{O(N)}, enquanto a abordagem utilizando a fórmula matemática para progressões 
aritméticas, resolve com complexidade \textbf{O(1)}.

\begin{lstlisting}[caption={Função soma com complexidade linear}, label={lst:somaOn}]
def soma(n: int) -> int:
    """
    Calcula a soma dos n primeiros números naturais, incluindo o 0.
    Implementado com estruturas de repetição.

    return:
        Número inteiro referente a soma dos números do intervalo.
    """
    total = 0
    
    for num in range(n):
        total += num

    return total
\end{lstlisting}

\begin{lstlisting}[caption={Função soma com complexidade constante}][label={lst:somaO1}]
def soma(n: int) -> int:
    """
    Calcula a soma dos primeiros n números naturais, incluindo o 0.
    Implementado com a fórmula matemática de progressão aritmética.

    return:
        Número inteiro referente a soma dos n primeiros números.
    """
    return (n * (n - 1)) / 2
\end{lstlisting}


Como demostrado acima, ter conhecimento dos fundamentos matemáticos que envolvem o problema, pode gerar soluções com 
complexidade menor. Por conta disso, antes de descrever sobre a implementação do código, é essencial estabelecer alguns 
conceitos fundamentais que orientaram na tomada de decisões dos algoritmos e das estruturas adotadas.
	